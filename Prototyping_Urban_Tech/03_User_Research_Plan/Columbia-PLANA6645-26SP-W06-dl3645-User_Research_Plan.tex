\documentclass[11pt]{article}
\usepackage[utf8]{inputenc}
\usepackage{geometry}
\geometry{a4paper, margin=1in}
\usepackage{titling}
\usepackage{hyperref}
\usepackage{parskip}
\usepackage{enumitem}
\usepackage{array}
\usepackage{booktabs}
\usepackage{amssymb}
\usepackage{needspace}

\title{User Research Plan \\ \large PLANA6645 -- Prototyping in Urban Tech}
\author{Daniel Lewis}
\date{February 24, 2026}

\begin{document}

\maketitle

\section*{Overview}

This document outlines a two-to-three-week user research plan for \textbf{Homecastr}, a probabilistic home price forecasting tool live at \href{https://homecastr.com}{homecastr.com}. The plan covers outreach methods, a research schedule, target participants, and interview and survey materials that will guide the discovery phase of the project.

Homecastr's go-to-market strategy sequences in two phases: consumer-first, where free forecasts attract homeowners and buyers at scale and generate the intent data that becomes the product's long-term data moat, followed by operator monetization, selling forecast access and intent signals to SFR operators and RE investors. This means homeowners and buyers are the primary focus for product-led growth and interface design, while operators and investors are the primary focus for the revenue-generating tier: and what data outputs must look like to be actionable.

This plan follows the course's User Research Principles \& Tactics, prioritizing transparency, accessible language, data privacy, and meaningful inclusion of the users most affected by information asymmetry in housing markets.

\section*{Types of Outreach}

Three complementary channels adapted to each user group:

\subsection*{1. Organizational Outreach (Primary)}

Direct contact with housing counseling nonprofits and real-estate professional organizations to recruit participants actively engaged in property decisions. The ask is either permission to attend a session and recruit participants afterward, or an introduction to recent program participants willing to speak for 20--30 minutes.

\subsection*{2. Snowball / Network Outreach (Secondary)}

Leveraging personal and academic networks to reach participants through warm introductions. Each interviewee will be asked at the end of their session: ``Is there anyone else you think I should talk to?''

\subsection*{3. Digital Survey Distribution (Supplementary)}

A short online survey distributed via direct links to organizational contacts, Columbia student listservs, Slack and Discord communities, LinkedIn, and relevant Reddit communities:

\begin{itemize}[leftmargin=2em]
    \item \mbox{r/RealEstate}
    \item \mbox{r/FirstTimeHomeBuyer}
    \item \mbox{r/realestateinvesting}
\end{itemize}

The survey broadens the sample and pre-screens potential interview candidates who indicate interest in a follow-up conversation.

\needspace{12\baselineskip}
\section*{Research Schedule}

\begin{center}
\begin{tabular}{@{} l >{\raggedright\arraybackslash}p{9.5cm} @{}}
\toprule
\textbf{Period} & \textbf{Activities} \\
\midrule
\textbf{Week 6} & \textbf{Preparation \& Initial Outreach} \\
Feb 24--28 & \begin{itemize}[nosep, leftmargin=*]
    \item Draft informed consent form
    \item Finalize interview scripts and survey instrument
    \item Send initial outreach emails to all identified contacts
    \item Launch online survey
    \item Schedule first interviews for Week 7
\end{itemize} \\
\midrule
\textbf{Week 7} & \textbf{Primary Research Execution} \\
Mar 3--7 & \begin{itemize}[nosep, leftmargin=*]
    \item Conduct 4--6 interviews across both user groups
    \item Follow up with non-respondents from initial outreach
    \item Begin snowball recruitment from early interviewees
    \item Promote survey; target 15--20 completions
\end{itemize} \\
\midrule
\textbf{Week 8} & \textbf{Continued Research \& Early Synthesis} \\
Mar 10--14 & \begin{itemize}[nosep, leftmargin=*]
    \item Conduct 3--5 additional interviews to close sample gaps
    \item Close survey
    \item Transcribe and code interview notes
    \item Begin affinity mapping and thematic analysis
    \item Draft preliminary insights for Research Synthesis
\end{itemize} \\
\bottomrule
\end{tabular}
\end{center}

\textbf{Target sample size:} 8--12 interviews total across both user groups, plus 15--20 survey responses: enough to identify consistent patterns across different user types.

\section*{Identified Participants and Groups}

\subsection*{User Group A: Homeowners, Buyers, and Sellers}

Individuals making personal residential real-estate decisions. Priority sub-segment: first-time homebuyers in high-variance markets, who face the highest stakes and the least informational support.

\textit{Houston:}
\begin{itemize}
    \item \textbf{Avenue CDC}: HUD-certified homebuyer education and one-on-one counseling
    \item \textbf{Houston Area Urban League (HAUL)}: First-time homebuyer workshops for low-to-moderate-income families
    \item \textbf{Fifth Ward CRC}: Homebuyer education in a historically underinvested neighborhood undergoing rapid change
\end{itemize}

\textit{New York:}
\begin{itemize}
    \item \textbf{NHSNYC}: Homebuyer counseling for New York City residents
    \item \textbf{Center for NYC Neighborhoods}: Advocates for affordable homeownership for middle- and working-class families
    \item \textbf{NYC Homeownership Network / LISC NY}: Counseling and financial coaching for aspiring homeowners
    \item \textbf{Columbia GSAPP classmates}: Accessible early interviewees navigating housing decisions in New York City
\end{itemize}

\subsection*{User Group B: Institutional Investors, Lenders, and Portfolio Managers}

Data-literate professionals: SFR operators, RE investors, and mortgage lenders: who require auditable, scenario-level forecasts for underwriting, portfolio monitoring, and risk surveillance. This is the revenue-generating user group.

\textit{New York (Columbia network):}
\begin{itemize}
    \item \textbf{CBS Real Estate Association}: Aspiring investors, lenders, and fund managers connected to the Paul Milstein Center
    \item \textbf{Columbia MSRED PropTech Club}: Students evaluating PropTech products from both technology and industry perspectives
    \item \textbf{MetaProp accelerator portfolio}: Peer PropTech founders including PropRise and Aren
\end{itemize}

\textit{Houston (firms identified by running our forecast model against public Houston property acquisition records):}
\begin{itemize}
    \item \textbf{Candlewood Homes}: 310 acquisitions in the backtest period; model outperformed in all four years
    \item \textbf{Sunrise Property Partners}: 104 acquisitions; model outperformed in all four years
    \item \textbf{Open House Texas Realty}: Active local investor; directly reachable
    \item \textbf{Fairport Ventures}: 120 acquisitions; local investment fund
\end{itemize}

Outreach hook for Houston entities: \textit{``We analyzed your Houston acquisition history. Here's what our model sees.''}

\section*{Research Materials}

\subsection*{Informed Consent Protocol}

Before each interview, participants will receive a brief consent form explaining:

\begin{itemize}
    \item The purpose of the research (academic coursework at Columbia GSAPP exploring how people engage with home price data)
    \item That participation is entirely voluntary and can be withdrawn at any time for any reason
    \item That interviews will be recorded only with explicit consent; participants may decline
    \item That all responses will be anonymized; no personal identifiers will appear in any written materials
    \item That all information shared will remain confidential within this academic project; there are no anticipated circumstances under which confidentiality would be broken
    \item How data will be stored and used (solely for this academic project, then deleted after course completion)
    \item The researcher's contact information for any follow-up questions
\end{itemize}

\subsection*{Semi-Structured Interview Script: Consumer Users}

\textbf{Duration:} 20--30 minutes \quad \textbf{Format:} Zoom or in-person

\textbf{Introduction} (2 min)

Thank you for taking the time to speak with me. My name is Daniel Lewis and I'm a graduate student at Columbia University. I'm working on a project exploring how people use data when making decisions about homes. This conversation will take about 20--30 minutes. There are no right or wrong answers: I'm interested in your honest experience. [Review consent form; confirm recording permission.]

\textbf{Part 1: Current Experience} (8--10 min)
\begin{enumerate}
    \item Can you tell me a little about your current housing situation: are you a homeowner, renting, or in the process of buying?
    \item When you last looked up information about a home's value, what tools or resources did you use?
    \item What questions did those tools answer well? What was left unanswered?
    \item How confident did you feel in the information you found? What made you trust or distrust it?
\end{enumerate}

\textbf{Part 2: Forward-Looking Needs} (8--10 min)
\begin{enumerate}[resume]
    \item When thinking about a property: buying, selling, or holding it: how much do you think about what might happen to its value in the future?
    \item If you could see a range of possible future values for a property: an optimistic, expected, and pessimistic scenario over the next four years: how would that change the way you approach the decision?
    \item What would make you trust a home-price forecast? What would you need to see to feel confident acting on it?
    \item Is there information about a property or neighborhood that you wish you could access but can't?
\end{enumerate}

\textbf{Part 3: Concept Reaction} (5 min)

\textit{[Introduce Homecastr:]} ``I've been building a tool called Homecastr. It shows a range of possible future values for any specific property: an optimistic, expected, and pessimistic scenario over the next four years: based on public data about the property, the neighborhood, and the broader market. You can look up any Houston-area address and see that forecast right now at homecastr.com. It's free.''

\begin{enumerate}[resume]
    \item Based on what I've described, is this something you'd use? Why or why not?
    \item At what point in your decision-making process would you want to use it?
\end{enumerate}

\textbf{Wrap-Up} (2 min)
\begin{enumerate}[resume]
    \item Is there anything else about your experience with home price information that I haven't asked about?
    \item Is there anyone else you think I should talk to about this?
\end{enumerate}

\subsection*{Semi-Structured Interview Script: Institutional Users}

\textbf{Duration:} 20--30 minutes \quad \textbf{Format:} Zoom or in-person

\textbf{Introduction} (2 min)

Thank you for making the time. I'm Daniel Lewis, a graduate student at Columbia GSAPP. I'm researching how real-estate professionals use data to evaluate properties and markets: specifically looking at where current tools fall short. This will take about 20--30 minutes. There are no right or wrong answers; I'm looking to understand your actual workflow, not pitch anything. [Review consent form; confirm recording permission.]

\textbf{Part 1: Current Workflow} (8--10 min)
\begin{enumerate}
    \item Can you walk me through how you currently evaluate a potential acquisition or underwrite a deal?
    \item What data sources do you rely on most heavily? How do you access them?
    \item Where in your workflow do you feel least confident about the data you're using? What would better data look like?
    \item How do you currently account for future price movements or market uncertainty?
\end{enumerate}

\textbf{Part 2: Forecast Needs} (8--10 min)
\begin{enumerate}[resume]
    \item If you had access to property-level probabilistic forecasts: a range of likely outcomes over a four-year horizon: how would you integrate them into your process?
    \item What level of geographic granularity matters most: individual property, block, ZIP code, or metro?
    \item What would you need to see to trust a third-party forecast model?
    \item How do you currently handle scenario analysis or stress testing? What tools do you use?
\end{enumerate}

\textbf{Part 3: Product Fit \& Intent Data} (5 min)

\textit{[Introduce Homecastr:]} ``We've built a tool that generates property-level probabilistic forecasts: optimistic, expected, and pessimistic scenarios over four years: for any property in the Houston metro, based entirely on public records. We're live at homecastr.com with over a million properties indexed. There's a free consumer tier and a paid operator tier planned at \$99/month for bulk access and API integration.''

\begin{enumerate}[resume]
    \item How does what I've described compare to what you currently use?
    \item Would you prefer a dashboard, an API, or exportable reports? Why?
    \item We also capture conversational signals from users asking about specific properties. Would aggregate buyer intent data: knowing which properties or neighborhoods are being actively researched: be useful in your workflow? How?
    \item What would be a dealbreaker that would prevent you from adopting a tool like this?
\end{enumerate}

\textbf{Wrap-Up} (2 min)
\begin{enumerate}[resume]
    \item Is there anything about your data or forecasting needs that I haven't asked about?
    \item Is there anyone else in your organization or network who would be worth speaking with?
\end{enumerate}

\subsection*{Online Survey Instrument}

\textbf{Duration:} 5--7 minutes \quad \textbf{Platform:} Google Forms

\vspace{8pt}
\textbf{Section 1: About You}

\textbf{1.} Which best describes you?

\begin{itemize}[label=$\square$, leftmargin=2em, itemsep=2pt]
    \item Homeowner
    \item Renter looking to buy
    \item Real estate investor
    \item Lender or mortgage professional
    \item Property manager
    \item Other: \underline{\hspace{5cm}}
\end{itemize}

\vspace{6pt}
\textbf{2.} In what metro area are you primarily looking at properties?

\underline{\hspace{10cm}}

\vspace{6pt}
\textbf{3.} How often do you look up home value information?

\begin{itemize}[label=$\square$, leftmargin=2em, itemsep=2pt]
    \item Weekly
    \item Monthly
    \item A few times a year
    \item Rarely
\end{itemize}

\vspace{8pt}
\textbf{Section 2: Current Tools and Pain Points}

\textbf{4.} Which tools do you use to research property values? \textit{(Select all that apply)}

\begin{itemize}[label=$\square$, leftmargin=2em, itemsep=2pt]
    \item Zillow
    \item Redfin
    \item Realtor.com
    \item CoreLogic
    \item HouseCanary
    \item PropStream
    \item County appraisal district website
    \item My real estate agent
    \item Other: \underline{\hspace{5cm}}
\end{itemize}

\vspace{6pt}
\textbf{5.} How satisfied are you with the tools you currently use?

\begin{center}
1 \quad 2 \quad 3 \quad 4 \quad 5 \\[2pt]
{\small Very unsatisfied \hfill Very satisfied}
\end{center}

\textbf{6.} What is the biggest gap in the property data you currently have access to?

\underline{\hspace{\linewidth}}

\underline{\hspace{\linewidth}}

\vspace{8pt}
\textbf{Section 3: Forecast Interest}

\textbf{7.} How valuable would it be to see a range of possible future values (optimistic, expected, pessimistic) for a specific property over the next four years?

\begin{center}
1 \quad 2 \quad 3 \quad 4 \quad 5 \\[2pt]
{\small Not at all valuable \hfill Extremely valuable}
\end{center}

\textbf{8.} What would make you trust a home-price forecast? \textit{(Select all that apply)}

\begin{itemize}[label=$\square$, leftmargin=2em, itemsep=2pt]
    \item A track record of accuracy
    \item Transparency about the methodology
    \item Endorsement by a known institution
    \item Ability to see the underlying data
    \item Comparison to other forecasts
    \item Other: \underline{\hspace{5cm}}
\end{itemize}

\vspace{6pt}
\textbf{9.} Would you be interested in a 20--30 minute follow-up interview?

\begin{itemize}[label=$\square$, leftmargin=2em, itemsep=2pt]
    \item Yes
    \item No
\end{itemize}

\textbf{10.} If yes, please share your email address so I can follow up: \textit{(optional)}

\underline{\hspace{10cm}}

\section*{Ethical Considerations}

Consistent with the course's User Research Principles \& Tactics, this plan incorporates the following safeguards:

\begin{description}[leftmargin=0.5cm]
    \item[\textbf{Transparency:}] All participants will receive a clear, jargon-free explanation of the research purpose, how their data will be used, and what outcomes they can and cannot expect.
    \item[\textbf{Power Shift:}] The plan prioritizes outreach to HUD-certified counseling organizations serving low-to-moderate-income and first-time homebuyers, centering the voices of those most affected by information asymmetry in housing markets.
    \item[\textbf{Data Privacy:}] Only directly relevant information will be collected. All responses will be anonymized. Interview recordings will be stored securely and deleted after transcription and course completion.
    \item[\textbf{Voluntary Participation:}] Participants can opt out at any time for any reason. Consent will be confirmed before each interview.
    \item[\textbf{Accessible Language:}] Scripts and survey questions avoid technical jargon. Probabilistic forecasts will be introduced using plain-language descriptions before any technical terms are used.
\end{description}

\end{document}

