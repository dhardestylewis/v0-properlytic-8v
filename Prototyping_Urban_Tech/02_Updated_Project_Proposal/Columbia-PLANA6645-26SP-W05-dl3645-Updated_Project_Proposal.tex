\documentclass[11pt]{article}
\usepackage[utf8]{inputenc}
\usepackage{geometry}
\geometry{a4paper, margin=1in}
\usepackage{titling}
\usepackage{hyperref}
\usepackage{parskip}
\usepackage{enumitem}

\title{Updated Project Proposal \\ \large PLANA6645 -- Prototyping in Urban Tech}
\author{Daniel Lewis}
\date{February 20, 2026}

\begin{document}

\maketitle

\section*{Project: Homecastr, Probabilistic Home Price Forecasting}

My project, called \textbf{Homecastr}, provides forward-looking, probabilistic home price forecasts that make market uncertainty legible at the property level. The core premise remains the same as my initial proposal: in heterogeneous markets, a single-point estimate like a Zestimate tells homeowners what their property may be worth \emph{today} but offers no view of where values are \emph{heading}. Homecastr fills this gap by generating scenario-based forecast bands, visualized as intuitive fan charts, that show a range of potential future valuations over a four-year horizon.

Over the past several weeks I have refined this vision in two important ways. First, the scope of the project has narrowed to focus on \emph{design process} rather than engineering finality. While I have a functional prototype already live at \href{https://homecastr.com}{homecastr.com} with over one million Houston-area properties indexed, the purpose of this course is to ensure the product I am building is genuinely driven by user needs and not only by technical capability. Second, I have begun to situate Homecastr within the broader landscape of residential real-estate data tools, which has sharpened how I articulate its value proposition and where it falls short.

The design processes I intend to follow include: (1)~\textbf{competitive analysis} to understand the ecosystem my users already navigate; (2)~\textbf{semi-structured user interviews} to surface unmet needs and gauge reactions to probabilistic forecasts; (3)~\textbf{iterative prototyping} in Figma, informed by interview insights, to evolve the dashboard UX; and (4)~\textbf{usability testing} with early adopters who can stress-test the interface and the mental model it demands.

\section*{Product: An Interactive Forecast Dashboard}

\subsection*{Current State}

The working prototype is an interactive, API-first dashboard built with Next.js and Supabase. Users can search for any Houston-area address and see: (a)~probabilistic forecast bands showing optimistic, expected, and pessimistic confidence intervals over four years; (b)~scenario analysis views that visualize alternative outcome paths generated by a diffusion world model; and (c)~explainable risk metrics including confidence scores, market volatility, and downside risk for specific micro-locations. The model is trained on over one million lots of public-record data and achieves a 14\% median forecast error across the Houston metro.

\subsection*{Competitor Analysis}

Following Professor Piacentini's recommendation, I have examined how existing tools fit into a typical user's experience and where gaps remain. The residential real-estate data landscape can be mapped along two axes: \emph{backward-looking vs.\@ forward-looking} and \emph{consumer vs.\@ institutional}.

\begin{description}[leftmargin=1.2cm, labelwidth=1.1cm, labelsep=0.1cm]
    \item[\textbf{\href{https://www.zillow.com}{Zillow} / \href{https://www.redfin.com}{Redfin}}] Consumer-facing point estimates of current value (Zestimate, Redfin Estimate) with no probabilistic range, scenario analysis, or forward-looking forecast~\cite{zillow_zestimate}. These tools define how millions of homeowners understand value but leave the ``what's next?'' question unanswered.
    \item[\textbf{\href{https://www.propstream.com}{PropStream}}] Comps, owner data, and lead lists used by individual investors for acquisition prospecting~\cite{propstream}. Powerful for deal-finding but entirely historical and transactional; no forward-looking valuation modeling.
    \item[\textbf{\href{https://www.corelogic.com}{CoreLogic}}] ZIP-level HPI forecasts and stress-testing scenarios, with some data available via a self-serve store~\cite{corelogic_avm}. However, forecasts report relative index movements at the ZIP level, not dollar-denominated property-level valuations.
    \item[\textbf{\href{https://www.housecanary.com}{HouseCanary}}] Property-level 36-month forecasts with prediction intervals; self-serve from \$19/mo, API from \$0.50/call~\cite{housecanary_analytics}. The closest competitor on features, but delivered as API endpoints and PDF reports for institutional analysts~\cite{freddiemac_housecanary}, not as an interactive consumer experience.
\end{description}

The key insight is that \textbf{no existing product renders forward-looking, property-level forecasts as an interactive consumer experience}. CoreLogic and HouseCanary forecast forward, but for institutional analysts; Zillow and Redfin are consumer-friendly but backward-looking. Homecastr sits in this gap: rendering dollar-denominated forecast bands (optimistic, expected, and pessimistic scenarios) as interactive fan charts that make uncertainty legible to a homeowner while remaining credible to a sophisticated investor.

\subsection*{Relevant Data}

The underlying data pipeline draws on several public and proprietary sources, and Professor Piacentini's feedback has prompted me to think more carefully about which data are most \emph{relevant to each user group}:

\begin{itemize}
    \item \textbf{Public records}: Deed transfers, building permits, tax assessments, and appraised market values from the Harris County Appraisal District~\cite{hcad}. These form the model's training backbone.
    \item \textbf{Land-use and development data}: Parcel-level land-use classifications from the City of Houston GIS portal~\cite{cohgis} and monthly residential building permit records from the City of Houston Open Data portal~\cite{cohopendata}. Houston lacks traditional zoning, so these datasets capture the development dynamics that drive valuation trajectories.
    \item \textbf{Census and demographic data}: Neighborhood-level socioeconomic indicators from the American Community Survey~\cite{acs} that contextualize market trends and support institutional portfolio risk analysis.
    \item \textbf{Market indicators}: Mortgage rates, listing counts, and days-on-market from FRED~\cite{fred}; price-to-rent ratios from the FHFA HPI~\cite{fhfa_hpi} and BLS rent CPI~\cite{bls_rent}. These macro signals feed the scenario analysis module.
\end{itemize}

These data sources are model inputs, not features surfaced directly to users. The consumer sees forecast bands; the institutional API provides the auditability layer that lets operators trace forecasts back to their underlying drivers. A key design question is how much of this provenance to expose and at what level of detail, something I plan to explore through user interviews.

\subsection*{Affordances and Expertise Levels}

One of the most important pieces of feedback from my initial proposal was to consider the expertise levels of my potential users and how the design itself can \emph{teach} users how to engage. In the context of probabilistic forecasts, this means the interface must help users understand what a forecast band \emph{means}: that the shaded region represents a range of likely outcomes, not a guarantee, without requiring a statistics background. I plan to explore affordances such as guided tooltips, contextual explainers (e.g., ``This forecast says there is a 10\% chance the value drops below \$X''), and progressive disclosure that reveals technical detail on demand. For institutional users, the affordance challenge is different: these users expect rigor and auditability, so the product must make it easy to export confidence intervals and trace forecast inputs back to source data.

\section*{People: User Segments and Research Plan}

\subsection*{User Group A: Homeowners, Buyers, and Sellers}

This segment includes anyone making a personal residential real-estate decision, whether buying, selling, refinancing, or simply monitoring the value of their home. These users are typically \emph{not} data-sophisticated; they want clear, actionable answers (``Is now a good time to sell?'', ``Am I overpaying?'') rather than raw model outputs. Their current experience is dominated by Zillow and Redfin, tools that provide backward-looking valuations and leave the ``what's next?'' question unanswered.

Within this broad group, I have identified a particular sub-segment of interest: \textbf{first-time homebuyers in high-variance markets} (e.g., rapidly appreciating or gentrifying neighborhoods). These users face the highest stakes and the least informational support. Because Homecastr is live in the Houston metro, I plan to recruit interviewees from both Houston-area organizations (where participants can interact with real local data) and New York City channels accessible through Columbia:

\textbf{Houston area:}
\begin{itemize}
    \item \textbf{\href{https://www.avenuecdc.org}{Avenue CDC}}~\cite{avenuecdc}: HUD-certified 8-hour homebuyer education courses and one-on-one counseling. Clients are actively purchasing in the Houston metro.
    \item \textbf{\href{https://www.haul.org}{Houston Area Urban League (HAUL)}}~\cite{haul}: HUD-approved first-time homebuyer workshops covering readiness, financing, budgeting, and credit for low-to-moderate-income families.
    \item \textbf{\href{https://www.fifthwardcrc.org}{Fifth Ward CRC}}~\cite{fifthwardcrc}: Homebuyer education in Houston's Fifth Ward, a historically underinvested neighborhood now experiencing rapid change and high valuation variance.
\end{itemize}

\textbf{New York City:}
\begin{itemize}
    \item \textbf{\href{https://nhsnyc.org}{NHSNYC}}~\cite{nhsnyc}: Homebuyer education and counseling nonprofit serving New York City residents navigating first-time purchases.
    \item \textbf{\href{https://cnycn.org}{Center for NYC Neighborhoods}}~\cite{cnycn}: Protects affordable homeownership for middle- and working-class families; counselors can speak to information gaps.
    \item \textbf{\href{https://www.nyc-worth.org}{NYC Homeownership Network}}~\cite{nychn}: LISC NY collaborative connecting aspiring homeowners with counseling and financial coaching.
    \item \textbf{Columbia GSAPP classmates}: Peers currently navigating housing decisions in New York City who can serve as accessible early interviewees.
\end{itemize}

\subsection*{User Group B: Institutional Investors, Lenders, and Portfolio Managers}

This segment includes single-family rental (SFR) operators, mortgage lenders, and real-estate fund managers who require auditable, scenario-level forecasts for underwriting, portfolio monitoring, and risk surveillance. These users are data-literate and comfortable with quantitative outputs, but they demand transparency about model assumptions and error margins. I have identified the following specific outreach targets:

\begin{itemize}
    \item \textbf{\href{https://cbsrea.com}{CBS Real Estate Association}} (New York): Student club at Columbia Business School with connections to the Paul Milstein Center for Real Estate~\cite{cbsrea}. Includes aspiring investors, lenders, and fund managers.
    \item \textbf{\href{https://www.arch.columbia.edu/clubs}{Columbia MSRED PropTech Club}} (New York): GSAPP student organization connecting PropTech with traditional real estate. Members evaluate products from both technology and industry perspectives.
    \item \textbf{\href{https://www.metaprop.com}{MetaProp} accelerator portfolio} (New York): Peer PropTech founders, including PropRise (Cohort~10; AI-powered acquisition insights) and Aren (Cohort~7; computer-vision property assessment)~\cite{metaprop}.
    \item \textbf{\href{https://www.greenresidential.com}{Green Residential}} (Houston): Decades-long track record managing SFR portfolios in the Houston market where Homecastr is live~\cite{green_residential}. Can speak to how forecasts integrate into acquisition and management workflows.
    \item \textbf{\href{https://emersonpropertymanagement.com}{Emerson Property Management}} (Houston): Manages 200+ SFR units across greater Houston~\cite{emerson_pm}. Hands-on operators facing cash-flow, vacancy, and property-tax volatility challenges.
\end{itemize}

Whether the institutional and consumer experiences should be \emph{segmented} into distinct interfaces or follow similar general UI/UX patterns is an open design question I plan to explore through comparative user interviews.

\subsection*{User Research Plan}




\textbf{Phase 1: Discovery Interviews} (Weeks 5--7). Semi-structured, 20--30 minute conversations with 4--6 participants from each user group, conducted via Zoom or in person, to surface needs, pain points, and current workflows.

Sample questions for homeowners and buyers:
\begin{enumerate}
    \item When you last looked up your home's value (or a home you were considering), what tool(s) did you use and what questions were left unanswered?
    \item If you could see a range of possible future values for a property, how would that change the way you think about buying, selling, or holding?
    \item What would make you trust or distrust a home-price forecast? What would you need to see in the interface to feel confident acting on it?
\end{enumerate}

Sample questions for institutional users:
\begin{enumerate}
    \item Walk me through how you currently evaluate a potential acquisition or underwrite a loan. What data sources do you rely on?
    \item Where in your workflow do you feel least confident about the data you're using? What would ``better data'' look like?
    \item If you had access to property-level probabilistic forecasts, how would you integrate them into your existing process?
\end{enumerate}

\textbf{Phase 2: Concept Testing} (Weeks 8--10). 15--20 minute usability sessions using Figma prototypes informed by Phase~1 insights, where participants interact with low-fidelity wireframes of the dashboard.

Sample tasks for homeowners and buyers:
\begin{enumerate}
    \item Look at this fan chart for a property you're interested in. In your own words, what is it telling you about the property's future value?
    \item Using the dashboard, decide whether now is a good time to sell your home. Walk me through how you would reach that decision.
    \item Something about this forecast seems uncertain. Where would you click or look to understand why?
\end{enumerate}

Sample tasks for institutional users:
\begin{enumerate}
    \item You are evaluating a potential SFR acquisition. Using the dashboard, identify the key risk factors for this property over the next four years.
    \item Export the forecast data for this property. Does the output contain what you would need for an underwriting memo?
    \item The model shows wide confidence intervals for a particular neighborhood. What additional information would you need to act on this forecast?
\end{enumerate}

{\tiny
\begin{thebibliography}{99}

\bibitem{avenuecdc}
Avenue Community Development Corporation. (n.d.). \textit{Homebuyer education and counseling.} Avenue CDC. Retrieved February~19, 2026, from \url{https://www.avenuecdc.org}

\bibitem{bls_rent}
Bureau of Labor Statistics. (n.d.). \textit{Consumer Price Index for All Urban Consumers: Rent of Primary Residence in U.S. City Average (CUUR0000SEHA).} FRED. Retrieved February~19, 2026, from \url{https://fred.stlouisfed.org/series/CUUR0000SEHA}

\bibitem{cnycn}
Center for New York City Neighborhoods. (n.d.). \textit{About us.} CNYCN. Retrieved February~19, 2026, from \url{https://cnycn.org/about}

\bibitem{cohgis}
City of Houston. (n.d.). \textit{COHGIS Data Hub.} City of Houston GIS. Retrieved February~19, 2026, from \url{https://cohgis-mycity.opendata.arcgis.com}

\bibitem{cohopendata}
City of Houston. (n.d.). \textit{City of Houston Open Data.} City of Houston. Retrieved February~19, 2026, from \url{https://data.houstontx.gov}

\bibitem{cbsrea}
Columbia Business School Real Estate Association. (n.d.). \textit{About the Real Estate Association.} CBS REA. Retrieved February~19, 2026, from \url{https://cbsrea.com}

\bibitem{corelogic_avm}
CoreLogic (now Cotality). (n.d.). \textit{About CoreLogic.} CoreLogic. Retrieved February~19, 2026, from \url{https://www.corelogic.com/about-us/}

\bibitem{emerson_pm}
Emerson Property Management, LLC. (n.d.). \textit{About Emerson Property Management.} Emerson Property Management. Retrieved February~19, 2026, from \url{https://emersonpropertymanagement.com}

\bibitem{fhfa_hpi}
Federal Housing Finance Agency. (n.d.). \textit{FHFA House Price Index.} FHFA. Retrieved February~19, 2026, from \url{https://www.fhfa.gov/data/hpi}

\bibitem{fred}
Federal Reserve Bank of St.~Louis. (n.d.). \textit{Federal Reserve Economic Data (FRED).} FRED. Retrieved February~19, 2026, from \url{https://fred.stlouisfed.org}

\bibitem{fifthwardcrc}
Fifth Ward Community Redevelopment Corporation. (n.d.). \textit{About Fifth Ward CRC.} Fifth Ward CRC. Retrieved February~19, 2026, from \url{https://www.fifthwardcrc.org}

\bibitem{freddiemac_housecanary}
Freddie Mac. (2023, September~1). \textit{Single-family rental market at a glance.} Freddie Mac Research. \url{https://www.freddiemac.com/research/insight/20230901-single-family-rental-market}

\bibitem{green_residential}
Green Residential Properties, LLC. (n.d.). \textit{Meet our team.} Green Residential. Retrieved February~19, 2026, from \url{https://www.greenresidential.com/meet-our-team/}

\bibitem{hcad}
Harris County Appraisal District. (2024). \textit{Property tax trends: Harris County.} HCAD. Retrieved February~19, 2026, from \url{https://harriscountypropertytaxtrends.com}

\bibitem{haul}
Houston Area Urban League. (n.d.). \textit{First Time Homebuyers Workshop.} HAUL. Retrieved February~19, 2026, from \url{https://www.haul.org}

\bibitem{housecanary_analytics}
HouseCanary, Inc. (n.d.). \textit{Data, analytics \& valuations.} HouseCanary. Retrieved February~19, 2026, from \url{https://www.housecanary.com/solutions/data-analytics-valuations}

\bibitem{nychn}
Local Initiatives Support Corporation New York. (n.d.). \textit{NYC Homeownership Network (NYCHN).} LISC NY. Retrieved February~19, 2026, from \url{https://www.lisc.org/new-york/our-impact/nychn/}

\bibitem{metaprop}
MetaProp. (2025). \textit{MetaProp Accelerator at Columbia University: Cohort 10.} MetaProp. \url{https://www.metaprop.com/accelerator}

\bibitem{nhsnyc}
Neighborhood Housing Services of New York City, Inc. (n.d.). \textit{Mission \& vision.} NHS NYC. Retrieved February~19, 2026, from \url{https://nhsnyc.org}

\bibitem{propstream}
PropStream, Inc. (n.d.). \textit{Real estate investing software: Find deals before your competition.} PropStream. Retrieved February~19, 2026, from \url{https://www.propstream.com}

\bibitem{acs}
U.S. Census Bureau. (n.d.). \textit{American Community Survey (ACS).} U.S. Census Bureau. Retrieved February~19, 2026, from \url{https://data.census.gov}

\bibitem{zillow_zestimate}
Zillow, Inc. (n.d.). \textit{What is a Zestimate?} Zillow. Retrieved February~19, 2026, from \url{https://www.zillow.com/z/zestimate/}

\end{thebibliography}
}

\end{document}
